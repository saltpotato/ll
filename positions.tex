
\cventry{ Juli 2018--heute }{ Softwareentwickler }{ Beissbarth GmbH }{ München }{}{ Fahrwerkvermessung, Bremsenprüfung, Prüfstrassen und Scheinwerfereinstellung \begin{itemize}
\item Weiterentwicklung für bestehende Bremsen Prüfsstandssoftware \myhref{http://www.icperform.com }{ICPerform}
\item Mitentwicklung ICPerform-Car für zweiachsige PKW
\item Entwicklung eines Zwischenreleases für ICperform-Truck
\item Betreuung der Anbindung bestehender \myhref{https://www.amandaequipment.com/en/equipment/break-and-chassis-equipment/break-service-equipment-accessories/brake-system-air-pressure-gauge-set-mrs-433 }{MRS-433} Druckluftsensoren bezüglich Wegfahralarm und Messstetigkeiten
\item Entwickeln der .net Schnittstelle für Druckluftsensoren neuer Generation
\item Gelegentlicher vor Ort Einsatz in den Werkstätten (Bayern, BW)    
\end{itemize} } 
\cventry{ April 2014--Juni 2018 }{ Softwareentwickler }{ Distec GmbH }{ Germering }{}{ Spezialist für LCD Flachbildschirm-Lösungen \begin{itemize}
\item VideoPoster Mediaplayer Anwendung \myhref{https://www.distec.de/service/downloads/mediaplayer-downloads/ }{ACC3}
\item \myhref{http://www.datadisplay-group.com/fileadmin/pdf/Finished_products/WEBPOSTER/Datasheet_WEB-Poster.pdf }{WebPoster-Firmware,} Debian Wheezy auf Jessie Portierung
\item \myhref{http://www.datadisplay-group.com/tft-controller/industrial-mediaplayers/videoposter-iii/ }{VideoPoster-III}
\item Weiterentwicklung der Modifikation \myhref{https://www.datadisplay-group.de/produkte/kundenspezifische-loesungen/referenzprojekte/ }{LED-Anzeigen} am Flughafen San Francisco
\item \myhref{http://www.datadisplay-group.com/tft-controller/tft-controller-usb-lan/artistausb-eco/ }{USB-Display} Treiber Desktop Extension/Cloning, zusammen mit \myhref{https://spacedesk.net/de/ }{Dienstleister}
\item C++ Tool zur Produktion des Raspberry Pi compute module boards \myhref{https://www.datadisplay-group.de/produkte/kitloesungen/mit-arm-raspberry-pi/ }{Artista-Iot}
\item Evaluation von Windows-Iot auf Artista IoT zur Verwendung als HMI-Platform
\item Erstellen einer QML \myhref{https://www.youtube.com/watch?v=tnY_23eRoyU }{Demo-Videos}
\item C# Tool für Luminanzkalibration und -profiling mit Luxmetern \myhref{https://www.xrite.com/de/categories/calibration-profiling/i1display-pro }{i1Display-Pro} und \myhref{https://gossen-photo.de/mavolux-5032-b-usb/ }{Gossen-Mavolux} unter Verwendung von \myhref{https://www.argyllcms.com/ }{Argyll-CMS}
\item Maintanace für \myhref{https://www.distec.de/produkte/tft-controller/konfigurations-software/ }{Chandler/Mars/SmartLed-Rover} und -Prep Apps
\item Golang Tool zur Boardkonfiguration mit RS232
\item Powershell Buildskripte für Debian-builds mit VirtualBox (vboxmanage)
\item Vereinfachter Kaffeeautomat-HMI mit \myhref{https://www.embedded-wizard.de/ }{Embedded-Wizard}
\item Umstellung Builds auf Symantec Extended Validation Code Signing
\item Umstellung der meisten Qt-Anwendungen von Qt4 auf Qt5
\item Ansprechpartner rund um Windows innerhalb der Entwicklunsabteilung    
\end{itemize} } 
\cventry{ Januar 2012--Oktober 2013 }{ Softwareentwickler }{ CreateCtrl AG }{ München }{}{ Software für TV, VoD und Radio \begin{itemize}
\item Entwicklung von \myhref{http://www.createctrl.com/createctrlsuite.html }{Programm,} Rundfunk- und Video-on-Demand-Planung (VoD)
\item Wartung eines MFC-UIs mit neuen Felder und Masken, Pl-Sql Im- und Export-Procedures
\item Entwicklung einer neuen VoD-Planung mit WPF, Spring.NET und NHibernate\begin{itemize}
\item Entwurf und Implementierung des klassenbasierten OR Mappings
\item NHibernate Caching-Optimierung
\item Implementierung der Suche, Export- und Middlewarefunktionen, Drag&Drop\end{itemize}
\item Testbereitstellung und Go-Live    
\end{itemize} } 
\cventry{ Juli 2005--Dezember 2011 }{ Softwareentwickler }{ ZIP Industrieplanung }{ München }{}{ Systeme zur Fabrik- und Logistikplanung \begin{itemize}
\item Konzeption Entwicklung der Qt-UI für \myhref{http://www.exu.de/web/de/malagav3 }{Malaga-V3}
\item Registerkarten in den \myhref{http://www.exu.de/web/de/features }{Planungsbereichen} Produkt, Prozess und Ressource\begin{itemize}
\item Entwicklung der Bereiche Administration, insbesondere GUI, Mehrsprachigkeit, Datenanbindung
\item Implementierung des Bibliothekskonzeptes
\item Anbindung des Clients an Bentley Microstation V8i über MFC und MDL
\item Layout-Kopierfunktion
\item Implementierung der Ableitung des \myhref{http://www.exu.de/web/images/stories/deu/malaga/img4.jpg }{Montagevorranggrafen} zum Prozess
\item Aktualisierung der GUI über Multiuser
\item Implementierung des Reportingkonzeptes
\item Auswertungsfunktionen nach Powerpoint
\item Auslastungs-, Fahrplan-, und Bestands- Diagramm, \myhref{http://www.exu.de/web/images/stories/deu/malaga/img5.jpg }{Prozesseditor}\end{itemize}
\item Bereitstellung von Builds und Pflege des Build-Systems    
\end{itemize} } 
\cventry{ November 2003--Juni 2005 }{ Softwareentwickler }{ WSW Software GmbH }{ Gauting (München) }{}{ Hoch konfigurierbare IT-Logistiklösungen auf Basis von SAP und .Net \begin{itemize}
\item Cliententwicklung basierend auf \myhref{http://www.wsw-software.de/jis/wsw-jit/ }{Just-in-Time-Standardsoftware}
\item Datenbankentwicklung über SQL-Server und angeschlossener Prozesse
\item Visualisierung von Prozessabläufen (z.B. über Ampel), - Teilesequenzierung
\item Implementierung eines mobilen .Net Clients für Stapler und einen Barcode-Scanner
\item Export Prozessdaten nach Word    
\end{itemize} } 
\cventry{ März 2003--Oktober 2003 }{ Freiberuflicher Softwareentwickler }{ Baumer Group (ehem. Massen Machine Vision) }{ Konstanz }{}{ Hersteller von Sensoren, Drehgebern, Messinstrumenten und Komponenten für die automatisierte Bildverarbeitung \begin{itemize}
\item Pflege der Qt-GUI mit ca. 20 Masken Umfang
\item Konvertierung Konfigurationsdaten von Ini auf Xml    
\end{itemize} } 